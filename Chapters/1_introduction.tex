\chapter{Introduction}
\section{Context}
Artificial intelligence (AI) has increasingly become an integral part of our life, having an undeniable impact on today’s society. 
AI was defined, for the first time, in 1955 at Darthmounth Research project as problem of \textit{"making a machine behave in ways that would be called intelligent if a human were so behaving"}\cite{kaplan2019siri}. 
Over the years, the focus has not been limited to the theoretical aspect alone. The rapid development of technologies, the increase in computing power, as well as the widespread presence of sensors have enabled the application of artificial intelligence as a support technology in fields ranging from industry, healthcare, and business to education. \cite{busnatu2022clinical}
Artificial intelligence is not only applied to these sectors but can also be found in common applications such as social media, digital assistants, recommendations, online searches, and facial recognition \cite{ref1}. These are just some of the applications we interact with in our daily lives.

\subsection{Large language models in real life}
The first voice assistants began to appear starting in 2011, when Apple introduced Siri on its new iPhone model: a voice assistant capable of conversing with users in natural language. Later, other assistants were introduced to the market, including Amazon Alexa (2014) and Google Assistant (2016) \cite{ref2}. These assistants generally have the same functionalities, being able to send messages and have simple conversations with users. However, starting in November 2022, the performance of these assistants has been surpassed by a new text-based assistant released by OpenAI: ChatGPT.\\
ChatGPT, in which GPT stands for \textit{Generative Pre-trained Transformer} a family of large language models created by OpenAI that uses deep learning to generate human-like, conversational text. 
ChatGPT represents a significant leap forward compared to the AI-based assistants available on the market until that time, as it can perform more advanced tasks than its competitors, including: writing a text/letter, coding, summarizing content, and writing Excel formulas, to name a few. Subsequent versions have integrated the DALL-E 3 model, capable of generating images, and have introduced GPT-4, the latest, more powerful, and up-to-date version of the large language model. \cite{ref3}

\subsection{Prompt engineering and the role of the prompt engineer}
The advent of advanced large language models like ChatGPT and BARD not only creates new opportunities for innovation and automation but also introduces significant challenges. One of the main challenges in using these models is creating and optimize prompts (questions posed to the model by the humans) \cite{ref5} that provide the model with the right instructions to generate accurate and relevant responses. Prompt engineering specifically addresses this challenge and focuses on defining the interactions and outputs of large language models, whose core purpose is to create optimal prompts for a generative model.\cite{amatriain2024prompt}
Everything lies in shaping the prompt while considering not only the user’s goal but also the context and the specific large language model being used, these aspects are considered by a new professional role that is emerging within companies: the prompt engineer. The prompt engineer must essentially select the most appropriate prompt engineering technique for a given task, a specific large language model, and the intended goal. The aim of this thesis is to provide a new tool to support their decisions: a prompt engineering ontology developed through an experimental approach.

\newpage
\section{Objectives}
This thesis sets out to bridge different domains: the prompt engineering domain, the large language models domain and the semantic web domain through the creation of an ontology within which state-of-the-art large language models and the most advanced prompt engineering techniques are represented.

\subsection{General Objectives}
\begin{itemize}
    \item \textbf{Investigating prompt engineering techniques:} The thesis includes an analysis of studies on the most recent prompt engineering techniques to represent in the ontology.

    \item \textbf{Analysis of large language model:} The thesis analyzes the state-of-the-art large language models that will be represented in the ontology.

    \item \textbf{Ontology study:} Study of ontologies and their application in similar fields.

    \item \textbf{Analysis of ontology engineering techniques:} The thesis analyzes various ontology engineering techniques with the aim of selecting the best approach for developing the ontology.
    
\end{itemize}


\subsection{Specific Objectives}
The specific objective of this thesis is to develop, using a state-of-the-art ontology engineering approach, a new ontology that represents prompt engineering techniques and large language models. The aim is to provide a useful tool that can be utilized not only by industry experts but also by students, content creators, and individuals without a specific background.